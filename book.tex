\documentclass{book}
\usepackage[paperwidth=60mm, paperheight=102.86mm, margin=0mm]{geometry}
\usepackage{tikz}
\usepackage{fontspec}

\newfontface\cardfont{EB Garamond SC}[LetterSpace=8]
\newfontface\reversedC{FreeSerif}
\newfontface\cuneiform{Noto Sans Cuneiform}
\setmainfont{Gentium Book Basic}[Scale=0.7]
\usepackage{leading}
\leading{8pt}

% Small page numbers at bottom center
\usepackage{fancyhdr}
\fancypagestyle{content}{%
  \fancyhf{}%
  \fancyfoot[C]{\fontsize{5}{6}\selectfont\thepage}%
  \renewcommand{\headrulewidth}{0pt}%
  \renewcommand{\footrulewidth}{0pt}%
}

% Smaller section headings for tiny pages
\usepackage{titlesec}
\titleformat{\chapter}[display]
  {\cardfont\fontsize{10}{12}\selectfont\centering}{}{0pt}{}
\renewcommand{\chaptername}{}
\newcommand{\numchapter}[2]{%
  \chapter*{{\fontsize{8}{10}\selectfont #1}\\[1mm] #2}%
  \addcontentsline{toc}{chapter}{#1\hspace{1.5mm}#2}%
}
\titleformat{\section}
  {\cardfont\fontsize{7}{9}\selectfont\centering}{}{0pt}{}
\titlespacing*{\chapter}{0pt}{5mm}{3mm}
\titlespacing*{\section}{0pt}{2mm}{1mm}

% Tiny TOC formatting
\usepackage{tocloft}
\renewcommand{\cftchapfont}{\cardfont\fontsize{6}{8}\selectfont}
\renewcommand{\cftchappagefont}{\fontsize{6}{8}\selectfont}
\renewcommand{\cftsecfont}{\fontsize{5}{7}\selectfont}
\renewcommand{\cftsecpagefont}{\fontsize{5}{7}\selectfont}
\renewcommand{\cfttoctitlefont}{\cardfont\fontsize{10}{12}\selectfont\centering}
\setlength{\cftbeforechapskip}{1mm}
\setlength{\cftbeforesecskip}{0.5mm}
\setlength{\cftsecindent}{3mm}
\renewcommand{\cftdotsep}{2}

\pagestyle{empty}

\begin{document}

% Title page
\noindent
\begin{tikzpicture}
  \useasboundingbox (0,0) rectangle (60mm, 102.86mm);
  \clip (0,0) rectangle (60mm, 102.86mm);
  % Radiating lines - black on white, matching card back pattern
  \foreach \a in {0,5,...,355} {
    \foreach \s in {0,1,...,19} {
      \pgfmathsetmacro{\op}{1 - \s/19}%
      \pgfmathsetmacro{\startrad}{5 + \s*2.5}%
      \draw[black, thin, opacity=\op]
        (30mm, 51.43mm) ++(\a:\startrad mm) -- ++(\a:2.5mm);
    }
  }
  % White backing to keep text readable
  \fill[white] (30mm, 51.43mm) circle (12mm);
  % Title text
  \node[anchor=center, font=\cardfont\fontsize{46}{50}\selectfont,
        text=black, text width=55mm, align=center]
    at (30mm, 51.43mm) {Star\\Taker\\Tarot};
\end{tikzpicture}

\newpage

% Page two - blank back of cover
\phantom{x}

\newpage

% Page three - author page
\noindent
\begin{tikzpicture}
  \useasboundingbox (0,0) rectangle (60mm, 102.86mm);
  % Title at top
  \node[anchor=north, font=\cardfont\fontsize{20}{24}\selectfont,
        text=black, text width=50mm, align=center]
    at (30mm, 97mm) {Star\\Taker\\Tarot};
  % "by" in the middle
  \node[anchor=center, font=\cardfont\fontsize{10}{14}\selectfont,
        text=black, text width=50mm, align=center]
    at (30mm, 51.43mm) {by};
  % Author and numeral at bottom
  \node[anchor=south, font=\cardfont\fontsize{10}{14}\selectfont,
        text=black, text width=50mm, align=center]
    at (30mm, 6mm) {Jes\\Annalemma\\Haze Janvir\\Dolan-Wolfe\\I{\reversedC Ↄ}{\reversedC Ↄ}CXCIX};
\end{tikzpicture}

\newpage

% Page four - dedications
\noindent
\begin{tikzpicture}
  \useasboundingbox (0,0) rectangle (60mm, 102.86mm);
  \node[anchor=center, font=\cardfont\fontsize{8}{12}\selectfont,
        text=black, text width=48mm, align=center]
    at (30mm, 51.43mm) {to {\cuneiform 𒀭𒈹}\kern1pt, Queen of Heaven\\[4mm]and to the muse who inspires\\all creation\\[4mm]and to my child Isaac\\who keeps my brain busy};
\end{tikzpicture}

% Switch to content layout with margins and page numbers
\newgeometry{paperwidth=60mm, paperheight=102.86mm, margin=4mm, top=6mm, bottom=8mm, footskip=4mm}
\pagestyle{content}
\fancypagestyle{plain}{\pagestyle{content}}
\setcounter{page}{1}

\tableofcontents

\numchapter{i}{Introduction}

\numchapter{0}{The Void}

{\fontsize{7.8}{8.8}\selectfont\itshape
Avalokiteshvara Bodhisattva, when deeply practicing prajña paramita, clearly saw that all five aggregates are empty and thus relieved all suffering. Shariputra, form does not differ from emptiness, emptiness does not differ from form. Form itself is emptiness, emptiness itself form. Sensations, perceptions, formations, and consciousness are also like this. Shariputra, all dharmas are marked by emptiness; they neither arise nor cease, are neither defiled nor pure, neither increase nor decrease. Therefore, given emptiness, there is no form, no sensation, no perception, no formation, no consciousness; no eyes, no ears, no nose, no tongue, no body, no mind; no sight, no sound, no smell, no taste, no touch, no object of mind; no realm of sight \ldots\ no realm of mind consciousness. There is neither ignorance nor extinction of ignorance\ldots\ neither old age and death, nor extinction of old age and death; no suffering, no cause, no cessation, no path; no knowledge and no attainment. With nothing to attain, a bodhisattva relies on prajña paramita, and thus the mind is without hindrance. Without hindrance, there is no fear. Far beyond all inverted views, one realizes nirvana. All buddhas of past, present, and future rely on prajña paramita and thereby attain unsurpassed, complete, perfect enlightenment. Therefore, know the prajña paramita as the great miraculous mantra, the great bright mantra, the supreme mantra, the incomparable mantra, which removes all suffering and is true, not false. Therefore we proclaim the prajña paramita mantra, the mantra that says: ``Gate Gate Paragate Parasamgate Bodhi Svaha.''

\begin{flushright}
--- Heart of Great Perfect Wisdom Sutra
\end{flushright}
}

\newpage

\noindent Let's start with nothing. No cards. Empty table. The void is the space of the reading. Some say that the void is Saturn, in his aspect as Kronos, time, who rules the domain that all things happen in (but we'll meet him later.) The sky is a void the planets move though, an empty hole that goes on forever. Perhaps the back of the card is the void, the Ein Sof (the Infinite), Ayin (the Nothingness), who precedes all things. Without nothing you cannot have anything. Without anything you cannot have something. Once you have something, well, now it's possible to have questions. Before we start, ask yourself: do you really want to know? do you feel lucky? or if not, can you love your fate anyway? Well, that's something.

\vfill
\begin{center}
\includegraphics[height=25mm]{cardback.pdf}
\end{center}
\vfill

\end{document}
